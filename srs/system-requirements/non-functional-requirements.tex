\subsubsection{User-friendly Interface}
  \label{system-requirements/non-functional/friendly-UI}
  The front-facing web application should have a user-friendly interface. It
  should be simple, easy to use, ergonomic, and so on. Users should be able to
  focus on working---not fiddling with the UI---while searching and querying the
  database.

  \begin{table}[h!]
    \caption{User-friendly Interface}
    \label{system-requirements/non-functional/friendly-UI-table}
    \begin{tabularx}{\textwidth}{|l|X|}
      \hline
      Title           & User-friendly User Interface. \\ \hline
      Source Scenario & None. \\ \hline
      Description     & User interface should be easy to use.\\ \hline
      Priority        & Medium: 3. \\ \hline
      Applicable FRs  & None. \\ \hline
    \end{tabularx}
  \end{table}

  UI should ideally be \textit{optimistic} and \textit{reactive}. Optimistic
  refers to performing changes locally---that is, assuming a permissible action
  was performed---before notifying the server. If the action should not have
  happened, it should be rolled back. Reactive refers to how the interface is
  built, i.e. in the style of React.js.

  This should not be difficult to accomplish, as Meteor and React.js easily
  allow for an optimistic and a reactive UI, respectively.

  \textbf{Relevant Links:}
  \begin{itemize}
    \item \url{https://www.meteor.com}
    \item \url{https://facebook.github.io/react/}
  \end{itemize}

  \begin{table}[h!]
    \caption{Optimistic User Interface}
    \label{system-requirements/non-functional/optimistic-UI-table}
    \begin{tabularx}{\textwidth}{|l|X|}
      \hline
      Title           & Optimistic User Interface. \\ \hline
      Description     & User interface should be optimistic.\\ \hline
      Priority        & Necessary: 0. \\ \hline
    \end{tabularx}
  \end{table}

  \begin{table}[h!]
    \caption{Reactive User Interface}
    \label{system-requirements/non-functional/reactive-UI-table}
    \begin{tabularx}{\textwidth}{|l|X|}
      \hline
      Title           & Reactive User Interface. \\ \hline
      Description     & User interface should be reactive.\\ \hline
      Priority        & Necessary: 0. \\ \hline
    \end{tabularx}
  \end{table}

\subsubsection{Scaleable database}
  Database must be scaleable. As mentioned in Section
  \ref{system-constraints/language/database}, MongoDB will be the database of
  choice. Fortunately, MongoDB is built to scale, so no work needs to be done.

  \begin{table}[h!]
    \caption{Scaleable database}
    \label{system-requirements/non-functional/scaleable-database-table}
    \begin{tabularx}{\textwidth}{|l|X|}
      \hline
      Title           & Scaleable database. \\ \hline
      % Source Scenario & None. \\ \hline
      Description     & Database must scale.\\ \hline
      Priority        & Low: 5. \\ \hline
      % Applicable FRs  & None. \\ \hline
    \end{tabularx}
  \end{table}

\subsubsection{Fast Search Time}
  Queries performed via the API and the front-facing web application need to be
  blazing fast. This non-functional requirement pairs well with the one
  mentioned in Section \ref{system-requirements/non-functional/friendly-UI}, as
  an optimistic and reactive UI is also one which is fast.

  \begin{table}[h!]
    \caption{Fast search}
    \label{system-requirements/non-functional/fast-search}
    \begin{tabularx}{\textwidth}{|l|X|}
      \hline
      Title           & Fast search. \\ \hline
      % Source Scenario & None. \\ \hline   % should be added at a later date
      Description     & Queries and searches should be fast.\\ \hline
      Priority        & Medium: 3. \\ \hline
      % Applicable FRs  & None. \\ \hline   % should be added at a later date
    \end{tabularx}
  \end{table}

\subsubsection{System Downtime}
  System Downtime should only happen during major updates and should take
 no more than 1 day to complete.

  \begin{table}[h!]
    \caption{System Downtime}
    \label{system-requirements/non-functional/system-downtime}
    \begin{tabularx}{\textwidth}{|l|X|}
      \hline
      Title           & System Downtime. \\ \hline
      % Source Scenario & None. \\ \hline   % should be added at a later date
      Description     & The only downtime that should be experienced would be
			 due to updating the software.\\ \hline
      Priority        & Medium: 3. \\ \hline
      % Applicable FRs  & None. \\ \hline   % should be added at a later date
    \end{tabularx}
  \end{table}

\subsubsection{Real Time Updates}
  Anytime the database is changed, the application must automatically
  account for the change in real time.


  \begin{table}[h!]
    \caption{Real Time Updates}
    \label{system-requirements/non-functional/real-time-updates}
    \begin{tabularx}{\textwidth}{|l|X|}
      \hline
      Title           & Real Time Updates. \\ \hline
      % Source Scenario & None. \\ \hline   % should be added at a later date
      Description     & Any uploaded images should be automatically added to
			 the database and can be queried in real time by any user.\\ \hline
      Priority        & High: 4. \\ \hline
      % Applicable FRs  & None. \\ \hline   % should be added at a later date
    \end{tabularx}
  \end{table}